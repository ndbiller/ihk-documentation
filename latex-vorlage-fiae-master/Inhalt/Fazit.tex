% !TEX root = ../Projektdokumentation.tex
\section{Fazit} 
\label{sec:Fazit}

\begin{quote}
    \textit{Gäbe es die letzte Minute nicht, so würde niemals etwas fertig.}\\
    Mark Twain, US-amerikanischer Schriftsteller
\end{quote}

\subsection{Soll-/Ist-Vergleich}
\label{sec:SollIstVergleich}
% Wurde das Projektziel erreicht und wenn nein, warum nicht?
% Ist der Auftraggeber mit dem Projektergebnis zufrieden und wenn nein, warum nicht?
% Wurde die Projektplanung (Zeit, Kosten, Personal, Sachmittel) eingehalten oder haben sich Abweichungen ergeben und wenn ja, warum?
% Hinweis: Die Projektplanung muss nicht strikt eingehalten werden. Vielmehr sind Abweichungen sogar als normal anzusehen. Sie müssen nur vernünftig begründet werden (\zB durch Änderungen an den Anforderungen, unter-/überschätzter Aufwand).
%\paragraph{Beispiel (verkürzt)}
% Tabelle~\ref{tab:Vergleich} Zeitplanung

An der folgenden Tabelle~\ref{tab:Vergleich} kann die Abweichung von der geplanten Projektzeit abgelesen werden. So wurde besonders in den Teilbereichen, bei denen die Datenübertragung zwischen den Systemen umgesetzt werden sollte mehr Zeit beansprucht, als geplant.

\tabelle{Soll-/Ist-Vergleich}{tab:Vergleich}{Zeitnachher.tex}

\subsection{Lessons Learned}
\label{sec:LessonsLearned}
% Was hat der Prüfling bei der Durchführung des Projekts gelernt (\zB Zeitplanung, Vorteile der eingesetzten Frameworks, Änderungen der Anforderungen)?

Die Ursachen für diese Zeitabweichung von der geplanten Zeit stellen auch eine der am nachhaltigsten gelernten Lektionen dar. Besonders schwierig sind Systeme zu testen, wenn man währenddessen immer wieder auf Antworten von Kollegen mit eigenem Zeitplan angewiesen ist. Deshalb sollte in solchen Situationen immer eine gewisse Pufferzeit für Abweichungen mit einkalkuliert werden. Auch ist die fertige Umsetzung des Projektes zu großen Teilen den im Vorfeld definierten Testfällen zu verdanken, die beim programmieren nach \ac{TDD}-Prinzipien bereits während der Erstellung eines Features mehr Freiheiten beim Refactoring von bestehender Logik verschaffen.

\subsection{Ausblick}
\label{sec:Ausblick}
%  Wie wird sich das Projekt in Zukunft weiterentwickeln (\zB geplante Erweiterungen)?

Sobald die benötigten Features umgesetzt sind und das Projekt im Produktivsystem zur Accounterstellung genutzt wird, werden neben evtl. Bugfixes wahrscheinlich weitere Funktionsanfragen aus den Abteilungen in das Formular integriert werden. Das setzen sonstiger buchbaren Optionen eines Vertrages im Account stellt eine bereits angedachte Erweiterungsmöglichkeit dar. Auch eine zusätzliche Erstellung der Einträge zu den Kunden in unserem internen Billing-System zur automatischen Integration in unseren Abrechnungsprozess ist denkbar.
