% !TEX root = ../Projektdokumentation.tex
\section{Fazit} 
\label{sec:Fazit}
Obwohl die Tests der Firewall am letzten Projekttag im Labor 3.1.01 nicht mehr rechtzeitig durchgeführt werden konnten, und die Erstellung der Dokumentation mit \LaTeX{ } sich als schwieriger und langwieriger als Angenommen darstellte, sind wir mit unserem Ergebnis durchaus zufrieden. Die Dokumentation ist noch zu umfangreich und Stellenweise nicht ganz ausgearbeitet, doch haben wir gerade durch sie einiges gelernt. Dies zeigt sich auch im folgenden Überblick des finalen Standes.

\subsection{Soll-/Ist-Vergleich}
\label{sec:SollIstVergleich}
% Wurde das Projektziel erreicht und wenn nein, warum nicht?
% Ist der Auftraggeber mit dem Projektergebnis zufrieden und wenn nein, warum nicht?
% Wurde die Projektplanung (Zeit, Kosten, Personal, Sachmittel) eingehalten oder haben sich Abweichungen ergeben und wenn ja, warum?
Durch ein Firewall-Script sowohl auf dem Inside- wie auch auf dem Outside-Router der \ac{DMZ} schließen wir zusätzliche Sicherheitslücken in unserem System. Und durch das nachträgliche Testen in einer mit Windows Server 2016 virtualisierten Netzwerkumgebung haben wir die im Projekt erlernten Fähigkeiten erfolgreich auf ein weiteres System portiert und so hoffentlich auch gleich gefestigt. So haben wir viel über Netzwerke, Firewall-Regeln, \ac{NAT} und Port-Forwarding mithilfe von iptables, Linux im allgemeinen, dessen grundsätzliche Verzeichnisstruktur, dem Arbeiten im Terminal, sowie zusätzlich den Umgang mit \LaTeX{} zur Anfertigung einer Projektdokumentation, die von der \ac{IHK} geforderten Richtlinien dazu, sowie dem Arbeiten mit virtualisierten Netzwerken gelernt. Leider konnten wir dies nicht ganz im Rahmen des gegebenen Zeitbudgets tun, daher wissen wir noch nicht, wie zufrieden unser Auftraggeber mit den erbrachten Leistungen sein wird, oder ob die Überschreitung der Abgabefrist sich als schlechte Schulnote widerspiegeln wird.
Uns ist durchaus bewusst, das in einem nicht akademischen Umfeld eine Verzögerung des Projektes zusätzliche Kosten bedeutet hätte. Allerdings sind wir auch der Meinung, eine entsprechend große Gegenleistung an Wissen und Erfahrung durch dieses Projekt gewonnen zu haben, um die Abweichung vom vorher geplanten Projektrahmen zu rechtfertigen.
% Hinweis: Die Projektplanung muss nicht strikt eingehalten werden. Vielmehr sind Abweichungen sogar als normal anzusehen. Sie müssen nur vernünftig begründet werden (\zB durch Änderungen an den Anforderungen, unter-/überschätzter Aufwand).
%\paragraph{Beispiel (verkürzt)}
Wie in Tabelle~\ref{tab:Vergleich} noch einmal genau zu erkennen ist, konnte die Zeitplanung bis auf wenige Ausnahmen, einige davon jedoch aus bereits unter \nameref{sec:Dokumentation} erwähnten Gründen mit gravierend abweichenden Zeiten, eingehalten werden (falls der Auftraggeber bei unserer verspäteten Abgabe nochmal beide Augen zudrückt).

\tabelle{Soll-/Ist-Vergleich}{tab:Vergleich}{Zeitnachher.tex}

\subsection{Lessons Learned}
\label{sec:LessonsLearned}
% Was hat der Prüfling bei der Durchführung des Projekts gelernt (\zB Zeitplanung, Vorteile der eingesetzten Frameworks, Änderungen der Anforderungen)?
Wir haben jedenfalls gelernt uns eine realistischere Zeitplanung für kommende Projekte zu erstellen und wissen nun auch, welch ein enormer Aufwand eine Dokumentation im Rahmen der \ac{IHK}-Richtlinien darstellt. Wir haben auch den Vorteil einer schon während der Arbeit vorhandenen Doku zu schätzen gelernt und haben nun sowohl unter Linux wie auch Windows einiges über die Konfiguration von Netzwerken und Firewalls sowie das Arbeiten mit virtuellen Netzen verinnerlicht.

\subsection{Ausblick}
\label{sec:Ausblick}
%  Wie wird sich das Projekt in Zukunft weiterentwickeln (\zB geplante Erweiterungen)?
Obwohl das Projekt beendet ist, können wir die virtualisierte Testumgebung benutzen, um weitere Übungen daran durchzuführen. So wollen wir zum Beispiel in Zukunft unser Testnetzwerk noch ausbauen und es um Domain-Controller, \ac{DHCP}-, \ac{DNS}-, \ac{FTP}- und Exchange-Server erweitern. Ein individueller, ausführlicherer Ausblick auf unsere weiteren Vorhaben kann jeweils unserem Kompetenzportfolio entnommen werden, welche im \Anhang{app:Kompetenz} zu finden sind.
