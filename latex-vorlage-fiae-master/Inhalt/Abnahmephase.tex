% !TEX root = ../Projektdokumentation.tex
\section{Abnahmephase} 
\label{sec:Abnahmephase}
Der Zugang zum Webserver ohne aktivierte Firewall konnte hier bereits zum Halbjahr bei Abnahme der Funktionalität des zugrundeliegenden Netzwerkes durch unseren Auftraggeber festgestellt werden. Eine \ac{HTML}-Seite mit Stand des aktuellen Projektfortschritts wurde mit Bootstrap selbst für mobile Endgeräte optimiert. Sie zeigte neben verschiedenen Gruppeninformationen auch den Netzplan und die vorläufige Dokumentation zusammen mit einer einfachen Liste aus roten und grünen Buttons für jede Projektanforderung. Somit war daraus einfach ersichtlich, welche der Aufgaben bereits erfüllt werden konnten.
Da es nach der nur bei einigen Gruppen stichprobenartig durchgeführten finalen Abnahme durch Herrn Henze nur noch die Abgabe der Dokumentation vor Ende des Projektes gibt, jedoch keinen real existierenden Kunden, bei dem die entworfene \ac{DMZ} umgesetzt werden soll, wird die Einführungsphase aus der weiteren Projektbeschreibung entfallen. Eine Beispielhafte Implementierung kann jedoch auch der \nameref{app:Test} entnommen werden.

\subsection*{Aufbau einer virtuellen Testumgebung: } Da die Originalmaschinen zum Testzeitpunkt nicht mehr verfügbar waren, wurde hierzu eine eigene Testumgebung mittels HyperV nachgestellt. Genauere Angaben über die Teststellung finden sich im zweiten Netzplan \ref{app:Netzplan}, den ausführlichen Testprotokollen \ref{app:Testprotokolle} und einer Dokumentation des virtuellen Testsystems \ref{app:Test}, alles zusammen zu finden im \Anhang{app:Test}.

\Zwischenstand{Abnahmephase}{Abnahme}
