% !TEX root = ../Projektdokumentation.tex
\section{Dokumentation}
\label{sec:Dokumentation}
% Wie wurde die Anwendung für die Benutzer/Administratoren/Entwickler dokumentiert (\zB Benutzerhandbuch, API-Dokumentation)?
Da unser Auftraggeber bereits früh im Projekt seine Vorliebe nach einer, den \ac{IHK}-Richtlinien entsprechend umgesetzten Dokumentation Ausdruck verlieh, beschlossen wir uns, seinem Wunsch zu entsprechen. So wurde die finale Dokumentation in \LaTeX{ } erstellt. Das Ergebnis mag sich zwar sehen lassen, dennoch schlägt die Bearbeitung der Dokumentation dank der aufgetretenen Schwierigkeiten im Umgang mit \LaTeX{} mit einem zu hohen Anteil des Zeitbudgets zu Buche. Nichtsdestotrotz hier das beschriebene Resultat. Wir hoffen, es war die Mühen wert. Zu Ihrer Erstellung wurden zusätzlich folgende Webseiten zu Hilfe gezogen: \cite{texstudio}, \cite{texbibtut} und vor allem \cite{manualdetailed}
% Hinweis: Je nach Zielgruppe gelten bestimmte Anforderungen für die Dokumentation (\zB keine IT-Fachbegriffe in einer Anwenderdokumentation verwenden, aber auf jeden Fall in einer Dokumentation für den IT-Bereich).

\paragraph*{Entwicklerdokumentation:}
Die der neben der Konfiguration angelegte Entwicklerdokumentation befindet sich im \Anhang{app:Anleitung}. Sie wurde als Schritt-für-Schritt-Anleitung zum Wiederherstellen des bereits erreichten Zustandes im Fall eines technischen Versagens geführt. Sie wurde basierend auf Informationen aus folgenden Webseiten erstellt: \cite{debianOrg12}, \cite{debianOrg5}, \cite{iptableshowto}, \cite{iptablestutorial} und \cite{nathowto} 
%Die Entwicklerdokumentation wurde mittels PHPDoc\footnote{Vgl. \cite{phpDoc}} automatisch generiert. Ein beispielhafter Auszug aus der Dokumentation einer Klasse findet sich im \Anhang{app:Doc}. 

\Zwischenstand{Dokumentation}{Dokumentation}
