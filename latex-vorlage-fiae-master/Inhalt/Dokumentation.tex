% !TEX root = ../Projektdokumentation.tex
\section{Dokumentation}
\label{sec:Dokumentation}
% Wie wurde die Anwendung für die Benutzer/Administratoren/Entwickler dokumentiert (\zB Benutzerhandbuch, API-Dokumentation)?
Die Projektdokumentation wurde, soweit dies möglich war, bereits während der Planung und Implementierung mit angelegt. Sie wurde in \LaTeX{ } erstellt, um den Anforderungen an die betriebliche Projektarbeit im Rahmen der Abschlussprüfung zum Anwendungsentwickler Seitens der \ac{IHK} gerecht zu werden. Für die Benutzerschulung wurde eigens eine Dokumentation der Funktionen im neuen Prozess im Firmenwiki erstellt, um auch bei zukünftigen Funktionsänderungen am Angebotssystem schnell erweiterbar zu sein.

% Hinweis: Je nach Zielgruppe gelten bestimmte Anforderungen für die Dokumentation (\zB keine IT-Fachbegriffe in einer Anwenderdokumentation verwenden, aber auf jeden Fall in einer Dokumentation für den IT-Bereich).

\subsection{Benutzerdokumentation}
\label{subsec:Benutzerdokumentation}
%Die Entwicklerdokumentation wurde mittels PHPDoc\footnote{Vgl. \cite{phpDoc}} automatisch generiert. Ein beispielhafter Auszug aus der Dokumentation einer Klasse findet sich im \Anhang{app:Doc}.

Die Benutzerdokumentation wurde zur Unterstützung der Mitarbeiter beim neuen Onboarding-Prozess im unternehmenseigenen Wiki erstellt und soll den Benutzern jederzeit einen einfachen Überblick über die Bedeutung der neuen Eingabefelder und den neuen Prozess im allgemeinen bieten. Sie wird zusammen mit der neuen Funktionalität während des Termines der geplanten Benutzerschulung vorgestellt. Ein Screenshot des Wiki-Eintrages befindet sich im ~\Anhang{subsec:Wiki}.

\Zwischenstand{Dokumentation}{Dokumentation}
