% !TEX root = ../Projektdokumentation.tex
\section{Entwurfsphase} 
\label{sec:Entwurfsphase}
% Erklärung
Da das Projekt die Erweiterung des Funktionsumfanges eines bereits bestehenden Systems darstellt, sind viele Entscheidungen zu den verwendeten Technologien bereits im Vorfeld getroffen. Im Folgenden wird deshalb detaillierter beschrieben, wie die geplanten Erweiterungen im existierenden Angebotssystem realisiert werden sollen.

\subsection{Zielplattform}
\label{sec:Zielplattform}
% Beschreibung der Kriterien zur Auswahl der Zielplattform (u.a. Programmiersprache, Datenbank, Client/Server, Hardware)

TODO: browser, html, css, javascript, ruby, rails, mongodb, aws, heroku, cloud

(TODO)

\subsection{Architekturdesign}
\label{sec:Architekturdesign}
% Netzpläne \Anhang{app:Netzplan}
% Beschreibung und Begründung der gewählten Anwendungsarchitektur (z.B. MVC). Ggfs. Bewertung und Auswahl von verwendeten Frameworks sowie ggfs. eine kurze Einführung in die Funktionsweise des verwendeten Frameworks.

TODO: rails, mvc

(TODO)

\subsection{Entwurf der Benutzungsoberfläche}
\label{sec:Benutzungsoberfläche}
% Entscheidung für die gewählte Benutzeroberfläche (z.B. GUI, Webinterface).
% Beschreibung des visuellen Entwurfs der konkreten Oberfläche (z.B. Mockups, Menüführung).
% Ggfs. Erläuterung von angewendeten Richtlinien zur Usability und Verweis auf Corporate Design.

TODO: für wen, bootstrap, screenshot, neue felder, mockup

(TODO)

\subsection{Datenmodell}
\label{sec:Datenmodell}
% Entwurf/Beschreibung der Datenstrukturen (z.B. ERM und/oder Tabellenmodell, XML-Schemas) mit kurzer Beschreibung der wichtigsten (!) verwendeten Entitäten.

TODO: erm contract, zustandsdiagramm contract

(TODO)

\subsection{Geschäftslogik}
\label{sec:Geschäftslogik}
% Modellierung und Beschreibung der wichtigsten (!) Bereiche der Geschäftslogik (z.B. mit Komponenten-, Klassen-, Sequenz-, Datenflussdiagramm, Programmablaufplan, Struktogramm, Ereignisgesteuerte Prozesskette (EPK)).
% Wie wird die erstellte Anwendung in den Arbeitsfluss des Unternehmens integriert?

TODO: mapping standard > pro, klassendiagramme standard und pro

(TODO)

\subsection{Maßnahmen zur Qualitätssicherung}
\label{sec:Qualitaetssicherung}
% Welche Maßnahmen werden ergriffen, um die Qualität des Projektergebnisses (siehe Kapitel~\ref{sec:Qualitaetsanforderungen}: \nameref{sec:Qualitaetsanforderungen}) zu sichern (\zB automatische Tests, Anwendertests)?
% \Ggfs Definition von Testfällen und deren Durchführung (durch Programme/Benutzer).

TODO: minitest, tdd, ci

(TODO)

\subsection{Pflichtenheft}
\label{sec:Pflichtenheft}
% Auszüge aus dem Pflichtenheft/Datenverarbeitungskonzept, wenn es im Rahmen des Projekts erstellt wurde.
% Refferenz:\nameref{sec:Lastenheft}
% Pflichtenheft: \Anhang{app:Pflichtenheft}
Im folgenden die wichtigsten vereinbarten Pflichten des Projektes bezüglich Qualitätssicherung, um die Funktionalität und Erweiterbarkeit während der aktiven Nutzung im Produktivsystem gewährleisten zu können.\\[1.5ex]
\textbf{Pflichten der Benutzungsoberfläche:}\\[1.5ex]
\textbf{PB10:} Die Daten des Übergabeprotokolls müssen persistiert, geladen und erneut persistiert werden.\\
\textbf{PB20:} Der Button zur Account-Erstellung muss den Datenversand über den AMQP Message-Queue-Bus auslösen.\\
\textbf{PB30:} Die Daten zur Account-Erstellung müssen persistiert, geladen und erneut persistiert werden.\\[1.5ex]
\textbf{Funktionelle Pflichten:}\\[1.5ex]
\textbf{PF10:} Für Benutzer mit der Rolle Manager muss der Button zur Account-Erstellung sowie die Formularfelder zur Account-Erstellung und das Übergabeprotokoll sichtbar sein. Für Benutzer mit der Rolle Verkäufer müssen die Formularfelder zur Account-Erstellung und das Übergabeprotokoll sichtbar sein. Für einen Benutzer ohne Rolle dürfen weder der Button zur Account-Erstellung noch die Formularfelder zur Account-Erstellung oder das Übergabeprotokoll vorhanden sein.\\
\textbf{PF20:} Bei einem Vertragsabschluss muss eine E-Mail an die Adresse onboarding@doctena.com gesendet werden.\\
\textbf{PF40:} Die Daten aus dem Vertrag müssen zur Account-Erstellung in das auf dem AMQP Message-Queue-Bus verwendete Datenformat übersetzt werden.\\
\textbf{PF50:} Bei mehr als einem Arzt pro Praxis muss ein zusätzlicher User als Admin-User ohne Arzt und Kalender erstellt und dieses zusätzliche Objekt über den Bus gesendet werden.\\

\Zwischenstand{Entwurfsphase}{Entwurf}
