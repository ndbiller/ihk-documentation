% !TEX root = Projektdokumentation.tex

% Es werden nur die Abkürzungen aufgelistet, die mit \ac definiert und auch benutzt wurden. 
%
% \acro{VERSIS}{Versicherungsinformationssystem\acroextra{ (Bestandsführungssystem)}}
% Ergibt in der Liste: VERSIS Versicherungsinformationssystem (Bestandsführungssystem)
% Im Text aber: \ac{VERSIS} -> Versicherungsinformationssystem (VERSIS)

% Hinweis: allgemein bekannte Abkürzungen wie z.B. bzw. u.a. müssen nicht ins Abkürzungsverzeichnis aufgenommen werden
% Hinweis: allgemein bekannte IT-Begriffe wie Datenbank oder Programmiersprache müssen nicht erläutert werden,
%          aber ggfs. Fachbegriffe aus der Domäne des Prüflings (z.B. Versicherung)

% Die Option (in den eckigen Klammern) enthält das längste Label oder
% einen Platzhalter der die Breite der linken Spalte bestimmt.
\begin{acronym}[WWWWWW]
    \acro{AMQP}{Advanced Message Queuing Protocol}
    \acro{AWS}{Amazon Web Services}
    \acro{CI}{Continuous Integration}
    \acro{CIO}{Chief Information Officer}
    \acro{CPP}{Central Patient Portal}
    \acro{CRM}{Customer Relationship Management}
    \acro{CRUD}{Create Read Update Delete}
    \acro{CTO}{Chief Technology Officer}
	\acro{JSON}{JavaScrip Open Notation}
    \acro{MVC}{Model View Controller}
    \acro{HAML}{\ac{HTML} abstraction markup language}
    \acro{HTML}{Hypertext Markup Language}
	\acro{IHK}{Industrie- und Handelskammer}
    \acro{TDD}{Test-Driven Development}
\end{acronym}
