\subsection{Lastenheft}
\label{app:Lastenheft}
Es folgt unser Lastenheft mit Fokus auf den Anforderungen:

Die Umsetzung muss folgende Anforderungen erfüllen: 
\begin{enumerate}[itemsep=0em,partopsep=0em,parsep=0em,topsep=0em]
\item DMZ
	\begin{enumerate}
	\item Die DMZ soll aus zwei virtuellen, zu Routern konfigurierten Linux-Distributionen bestehen, welch die Netze INSIDE, OUTSIDE und das DMZ-Netz miteinander verbinden. 
	\item Die Router sollen entsprechend des Netzplanes eingerichtet und konfiguriert werden.
	\item Die DMZ soll Zugriffe auf den Webserver erlauben, aber Zugriffe auf das INSIDE-Netz verhindern. Hierzu soll auf dem Outside-Router NAT, Portforwarding und eine Firewall laufen.
    \item Die Router sollen nur vom Client-Rechner her fernadministrierbar sein.
	\end{enumerate}
\item Client-Rechner
\begin{enumerate}
    \item Der Client-Rechner im INSIDE-Netz nutzt das Betriebssystem Windows.
    \item Der Webserver soll eine Webseite mit dem aktuellen Stand der Gruppe anzeigen.    
\end{enumerate}
\item Webserver
\begin{enumerate}
    \item Der Webserver nutzt das Betriebssystem Windows. Er wird über das Tool Mini-Webserver vom Auftraggeber bereitgestellt.
    \item Der Webserver im DMZ-Netz muss vom OUTSIDE-Netz über Port 80 erreichbar sein. Hierzu soll auf dem Outside-Router NAT und Port-Forwarding eingerichtet werden.
    \item Der Webserver soll eine Webseite mit dem aktuellen Stand der Gruppe anzeigen.    
\end{enumerate}
\item Firewall
\begin{enumerate}
    \item Die Firewall soll den Webserver in der DMZ über Port 80 erreichbar sein lassen.
    \item Die Firewall soll SSH nur vom Admin-PC zulassen.
    \item Die Firewall soll ICMP zulassen.
    \item Die Firewall soll DNS zulassen.
    \item Die Firewall soll RDP zulassen.
    \item Die Firewall soll per Script an- und ausschaltbar sein. Hierzu muss an diversen Stellen per Script die Linux-Systemkonfiguration verändert werden
\end{enumerate}
\item Sonstige Anforderungen
	\begin{enumerate}
	\item Das Projekt soll unter Berücksichtigung der von der IHK ausgegebenen Richtlinien für eine Projektdokumentation dokumentiert werden.
    \item Es soll ein logischer Netzplan in Papierform erstellt und der Dokumentation angefügt werden.
	\item Pro Person soll ein ausführliches Kompetenzportfolio erstellt werden, welches einen kritischen Überblick über unsere individuellen Kompetenzstände vor, während und nach dem Projekt liefert. Diese sollen der Dokumentation angehängt werden.
	\item Die Funktionalität der Firewall soll getestet und die Ergebnisse in zwei Testprotokollen festgehalten werden. Diese sind der Dokumentation anzuhängen.
	\end{enumerate}
\end{enumerate}

